Seja MM o problema da multiplicar duas matrizes quadradas: uma instância $I_\text{MM} = (A,B,n)$ de MM consiste de duas matrizes quadradas $A$ e $B$ de ordem $n$ e a saída é a matriz $C = A \times B$.

Seja MQ o problema de calcular o quadrado de uma matriz quadrada: uma instância $I_\text{MQ} = (P,n)$ de MQ consiste de uma matriz quadrada P de ordem $n$ e a saída é a matriz $Q = P \times P$.

Suponha que MM tenha cota inferior $\Omega\left(n^{2.5}\right)$. Mostre que MQ tem cota inferior $\Omega\left(n^{2.5}\right)$. Descreva claramente as funções $\tau_I$ e $\tau_S$. Justifique.

\itemdsep

\begin{lemma} \label{lemma:redutivel}
    O problema MM é redutível a MQ em tempo $O\left(n^2\right)$.
\end{lemma}

\begin{proof}
    Considere uma transformação $\tau_I$ de uma instância $I_\text{MM}$ para $I_\text{MQ}$ dada por $\tau_I(A, B, n) = (P, 2n)$ onde
    \[
        P_{2n \times 2n} = \begin{bmatrix} & A \\ B & \end{bmatrix}
        = \begin{bmatrix}
            0         &        & 0         & (A)_{1,1} & \dots  & (A)_{1,n} \\
                      & \ddots &           & \vdots    & \ddots & \vdots    \\
            0         &        & 0         & (A)_{n,1} & \dots  & (A)_{n,n} \\
            (B)_{1,1} & \dots  & (B)_{1,n} & 0         &        & 0         \\
            \vdots    & \ddots & \vdots    &           & \ddots &           \\
            (B)_{n,1} & \dots  & (B)_{n,n} & 0         &        & 0         \\
        \end{bmatrix}
    \]
    Mais especificamente,
    \[
        (P)_{i,j} = \begin{cases}
            ~ (A)_{i,j-n} & \text{se $1 \leq i \leq n$ e $n < j \leq 2 n$} \\
            ~ (B)_{i-n,j} & \text{se $n < i \leq 2 n$ e $1 \leq j \leq n$}\\
            ~ 0 & \text{caso contrário}
        \end{cases}
    \]

    Então, uma solução $Q_{2n \times 2n} = P \cdot P$ de MQ será dada por
    \[
        Q = P P
        = \begin{bmatrix} & A \\ B & \end{bmatrix} \begin{bmatrix} & A \\ B & \end{bmatrix}
        = \begin{bmatrix} A B & \\ & A B \end{bmatrix}
    \]

    E podemos tomar $\tau_S(Q) = C$ com $C_{n \times n}$ sendo as primeiras $n$ colunas das primeiras $n$ linhas de $Q$. Desse modo, temos uma solução $C = A \cdot B$ para MM pois,
    \begin{align*}
        (C)_{i,j} &= (Q)_{i,j} \\
            &= \sum_{k = 1}^{2n} (P)_{i,k} (P)_{k,j} \\
            &= \sum_{k = 1}^{n} (P)_{i,k} (P)_{k,j} + \sum_{k = n + 1}^{2n} (P)_{i,k} (P)_{k,j} \\
            &= \sum_{k = 1}^{n} 0 + \sum_{k = n + 1}^{2 n} (A)_{i,k-n} (B)_{k-n,j} \\
            &= \sum_{k = 1}^{n} (A)_{i,k} (B)_{k,j} \\
            &= (A B)_{i,j}
    \end{align*}

    Tanto $\tau_I$ quanto $\tau_S$ fazem apenas a criação de uma nova matriz e, dependendo da representação das matrizes, cópias dos elementos. Portanto, o tempo de execução pode depender no máximo da quantidade de elementos na matriz, fazendo com que $T_{\tau_I}(n) \in O\left(\left(2n\right)^2\right) = O\left(n^2\right)$ e $T_{\tau_S}(n) \in O\left(n^2\right)$. Logo, a redução do problema tem tempo $O\left(n^2\right)$.
\end{proof}

\begin{theorem}
    Se MM tem cota inferior $\Omega\left(n^{2.5}\right)$, então MQ também tem cota inferior $\Omega\left(n^{2.5}\right)$.
\end{theorem}

\begin{proof}
    Seja $\pi_\text{MM}$ com tempo $\Theta\left(n^{2.5}\right)$ uma solução ótima de MM e considere que MQ tem uma cota inferior $\Omega(f(n))$ (?)
\end{proof}
